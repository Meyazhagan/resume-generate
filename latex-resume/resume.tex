% -------------- Resume ---------------
\documentclass{resume}

\begin{document}

% --------- Contact Information -----------
\introduction[
    fullname={Meyazhagan C N},
    email={meyazhagan.ofcl@gmail.com},
    phone={+91 86672 83263},
    linkedin={linkedin.com/in/meyazhagan-cn},
    github={github.com/meyazhagan}
]

% --------- Open Source Contribution -----------
\begin{resumeSection}{Open Source Contribution}
    {\projectTitle[
        project={Amplication},
        projectUrl={github.com/amplication},
        technology={Typescript, React, Nest.js, AST, Docker, GraphQL, GRPC, Postgres}
    ]}
    {\openSourceItem[
        title={Enhancement: Generate dotenv keys sorted alphabetically},
        pr={github.com/amplication/amplication/pull/7564},
        description={
            \begin{itemize}
                \itemsep -6pt {}
                \item \textbf{Developed} a function to alphabetically sort environment keys.
                \item Reduced conflicts during merge operations.
            \end{itemize}
        }
    ]}
    {\openSourceItem[
        title={Feature: RabbitMQ Plugin},
        pr={github.com/amplication/plugins/pull/187},
        description={
            \begin{itemize}
                \itemsep -6pt {}
                \item \textbf{Developed} a RabbitMQ plugin to generate controllers and modules associated with RabbitMQ.
                \item Empowers users to \textbf{effortlessly} produce and consume events using RabbitMQ.
            \end{itemize}
        }
    ]}
    {\projectTitle[
        project={Meshery},
        projectUrl={github.com/meshery},
        technology={Go, Typescript, React, Kubernetes, Service mesh, Docker, SQLite}
    ]}
    {\openSourceItem[
        title={Feature: Support tailing logs for Kubernetes},
        pr={github.com/meshery/meshery/pull/7801},
        description={
            \begin{itemize}
                \itemsep -6pt {}
                \item Implemented log tailing functionality for Kubernetes logs.
                \item Enabled real-time log \textbf{monitoring} of Kubernetes deployments when the follow flag is enabled.
                \item \textbf{Enhanced} user experience in monitoring deployment logs.
            \end{itemize}
        }
    ]}
    {\openSourceItem[
        title={Feature: Database Summary Addition},
        pr={github.com/meshery/meshery/pull/7686},
        description={
            \begin{itemize}
                \itemsep -6pt {}
                \item Developed UI to display tables and its record counts in the reset page.
                \item \textbf{Collaborated} with a peer programmer to enhance the UI.
                \item Provided users with the ability to \textbf{monitor} database information.
            \end{itemize}
        }
    ]}
    {\projectTitle[
        project={Datree},
        projectUrl={github.com/datreeio},
        technology={Go, Kubernetes, YAML}
    ]}
     {\openSourceItem[
        title={Enhancement: add schema location and policy Configuration},
        pr={github.com/datreeio/datree/pull/927},
        description={
            \begin{itemize}
                \itemsep -6pt {}
                \item Introduced a feature to configure schema location and policy config into a \textbf{configuration file}.
                \item \textbf{Enhanced} customization and user experience for CLI tool users.
            \end{itemize}
        }
    ]}
    {\openSourceItem[
        title={Support: Exclude paths flag for datree test},
        pr={github.com/datreeio/datree/pull/932},
        description={
            \begin{itemize}
                \itemsep -6pt {}
               \item \textbf{Implemented} exclude paths flag for \textit{datree test} command.
\item Allowed users to specify specific paths or files to exclude from the testing process, \textbf{enhancing} tool flexibility.
            \end{itemize}
        }
    ]}
\end{resumeSection}

% --------- Work Experience -----------
\begin{resumeSection}{Work Experience}
    {\projectTitle[
        project={Iviva, Eutech Cybernetics, India},
        projectUrl={www.iviva.com/},
        technology={Jan 2022 - present}
    ]}
    {\experienceItem[
        position={Software Developer},
        technology={C\#, .NET, Typescript, Python, React, SCSS, AWS, SQL Server, Postgres, MongoDB},
        description={
            \begin{itemize}
                \itemsep -6pt {}
                \item \textbf{Developed} backend functionality across service, web, and data storage layers to ensure robust system performance.
                \item Researched and prototyped new technologies to assess suitability for \textbf{adoption} within projects.
                \item \textbf{Integrated} third-party solutions judiciously to enhance project capabilities and efficiency.
                \item Created \textbf{tooling} for infrastructure management and provided technical support for sales and marketing initiatives.
            \end{itemize}
        }
    ]}
\end{resumeSection}

% --------- Personal Projects -----------
\begin{resumeSection}{Personal Projects}
    {\openSourceItem[
        title={Ticketing System for Query Resolving},
        pr={github.com/Meyazhagan/ticketing-system-backend},
        description={
            \begin{itemize}
                \itemsep -6pt {}
                \item \textbf{Developed} a query resolving platform tailored for mentors and students.
                \item Implemented real-time status updates on queries and integrated a \textbf{real-time} chat feature.
            \end{itemize}
        }
    ]}
\end{resumeSection}

% --------- Skills -----------
\begin{resumeSection} {Skills} 
    \skillSection[]
\end{resumeSection}

% --------- Education -----------
% \vspace{-1.5em}
\begin{resumeSection}{Education}
    \educationItem[
        college={PSG College of Arts \& Science},
        duration={2016 - 2019},
        degree={B. Sc Electronics},
        cgpa={7.1}
    ]
\end{resumeSection}

\end{document}
